
\documentclass[spanish,a4paper,12pt]{article}

\usepackage{latexsym,amsfonts,amssymb,amstext,amsthm,amsmath}
\usepackage[spanish]{babel} %Corta palabras en español
\usepackage[utf8]{inputenc} %Escribir con acentos, ñ...
\usepackage[dvips]{epsfig}
\usepackage{doc}
\usepackage{graphicx} %inclusion de graficos

\begin{document}
\title{\bf Informe sobre $\pi$ usando \LaTeX{}}
\author{Cinthia Hernandez Trujillo}
\date{11 de abril de 2014}
\maketitle

\begin{abstract}
En esta práctica, vamos a definir de manera sencilla, los pasos que hay que seguir para el cálculo de pi en una máquina de cómputo. Para ello, vamos 
\end{abstract}

\section{Objetivos}
A lo largo de la historia, muchos personajes han intentado averiguar el mayor número de decimales posibles de pi. Ahora con la computación, podemos calcular
de una manera más fácil y rápida mayor decimales con una aproximación bastante aceptable.

\subsection{Formula de aproximación de pi}
La fórmula de aproximación de pi \footnote{la fórmula se calcula en cadena} viene dado por: 

\begin{equation}
   pi=1/n \sum_{i=1}^n f(x) 
   \label{eq:pi}
\end{equation}
donde f(x) viene definido por: 
\[
f(x)=\frac{4}{(1+x)^2}
\]
a la vez, x se calcula de la siguiente manera:
\[
x=\frac{i- 1/2}{n}
\]
para i=1,2,...,n. A medida que aumentamos "i" la aproximación se acerca cada vez más al valor real de pi.

\subsection{Explicacion de la formula}
Para cada valor de i, debemos calcular x, al tener el valor de x, nos vamos a f(x), y seguidamente vamos a la ecuacion \ref{eq:pi} que nos da una aproximación de pi.  

\section{seccion 2}

\subsection{Tabla}
\begin{tabular}{lrc}
    Intervalos & Subintervalo & Valor aproximado pi \\
    i=1 & [0,1] & 3.2 \\
    i=10 & [0,0.1], [0.1,0.2],... [0.9,1] &  3.142425985 \\
    i=100 & [0,0.01],... [0.99,1] & 3.14160098692 \\
\end{tabular}

Y así podemos ver como el valor aproximado de pi, a medida que aumenta el intervalo, se acerca al valor real de pi.
La ecuación \ref{eq:pi} la hemos cogido del documento pdf de la práctica 5, y los resultados de la tabla \cite{tabular} los hemos calculado
al ejecutar la práctica 5.

\subsection{Imagen}
\includegraphics[scale=0.3] {imagen1.eps}

\begin{thebibliography}{1}
\bibitem{latex} Revista Iberoamericana de Matemáticas http://rmi.rsme.es/
\bibitem{latex} Paquetes de LaTeX http://www.ciencia-explicada.com/2012/12/paquetes-de-latex-mi-lista-de.html
\bibitem{latex} Inserción de imágenes en LaTeX http://navarroj.com/latex/figuras.html
\end{thebibliography} {1}

\end{document}
